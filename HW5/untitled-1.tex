\documentclass{article}
\usepackage[utf8]{inputenc}
\usepackage{kotex}
\usepackage{amsmath}
\usepackage{kmath}
\usepackage{graphicx}
\usepackage{listings}
\usepackage{subfigure}
\usepackage[a4paper,left=26mm,right=26mm,top=27mm,bottom=27mm]{geometry}


\title{Programming Language hw5}
\author{B811056 노이진}
\date{May 2021}

\begin{document}

\maketitle

\section{hanoi}
과제에 대한 설명은 코드 내에 주석으로 적었습니다.
\subsection{code}
\begin{lstlisting}[escapeinside=``]
hanoi(X) :- move(X,1,2,3).
%` 하노이를 호출하면 move함수를 호출.`

move(0,_,_,_) :- !.
%` N이 1일 경우 M이 0이 되며 이때는 옮길 원판이 없다는 의미가(한개만 있을 경우 옮길 필요x) 되어 재귀를 멈춰줘야하므로 !를 이용해 cut.`
move(N,X,Y,Z) :-
	M is N-1,
	move(M,X,Z,Y),  % `N-1개의 원판을 X번에서 Z번으로 옮긴다`
	print_h(X,Y,N), % `N번째 원반을 목적지로 옮긴다(여기서 print를 해줘서 표현. 
	                    재귀가 되면 N은 N-1이 되므로 N이 0이 되면 cut이 된 후 순차적으로 print될 것이다.)`
	move(M,Z,Y,X).  % `다시 N-1개의 원판을 Z번에서 Y번으로 옮긴다`
	
print_h(X,Y,N) :-
	write(N), write(' -> ['), write(X), write(','), write(Y), write(']'), nl.
% `재귀의 레벨이 되는 N이 몇번째 board를 움직이는지 나타낼 수 있으므로 그대로 write을 이용해 print해준다.`
\end{lstlisting}
\subsection{trace}
\begin{lstlisting}[escapeinside=``]
?- trace.
true.

[trace]  ?- hanoi(3).
   Call: (10) hanoi(3) ? creep
   Call: (11) move(3, 1, 2, 3) ? creep
   Call: (12) _7462 is 3+ -1 ? creep
   Exit: (12) 2 is 3+ -1 ? creep
   Call: (12) move(2, 1, 3, 2) ? creep
   Call: (13) _7600 is 2+ -1 ? creep
   Exit: (13) 1 is 2+ -1 ? creep
   Call: (13) move(1, 1, 2, 3) ? creep
   Call: (14) _7738 is 1+ -1 ? creep
   Exit: (14) 0 is 1+ -1 ? creep
   Call: (14) move(0, 1, 3, 2) ? creep
   Exit: (14) move(0, 1, 3, 2) ? creep
   Call: (14) print_h(1, 2, 1) ? creep
   Call: (15) write(1) ? creep
1
   Exit: (15) write(1) ? creep
   Call: (15) write(' -> [') ? creep
 -> [
   Exit: (15) write(' -> [') ? creep
   Call: (15) write(1) ? creep
1
   Exit: (15) write(1) ? creep
   Call: (15) write(',') ? creep
,
   Exit: (15) write(',') ? creep
   Call: (15) write(2) ? creep
2
   Exit: (15) write(2) ? creep
   Call: (15) write(']') ? creep
]
   Exit: (15) write(']') ? creep
   Call: (15) nl ? creep

   Exit: (15) nl ? creep
   Exit: (14) print_h(1, 2, 1) ? creep
   Call: (14) move(0, 3, 2, 1) ? creep
   Exit: (14) move(0, 3, 2, 1) ? creep
   Exit: (13) move(1, 1, 2, 3) ? creep
   Call: (13) print_h(1, 3, 2) ? creep
   Call: (14) write(2) ? creep
2
   Exit: (14) write(2) ? creep
   Call: (14) write(' -> [') ? creep
 -> [
   Exit: (14) write(' -> [') ? creep
   Call: (14) write(1) ? creep
1
   Exit: (14) write(1) ? creep
   Call: (14) write(',') ? creep
,
   Exit: (14) write(',') ? creep
   Call: (14) write(3) ? creep
3
   Exit: (14) write(3) ? creep
   Call: (14) write(']') ? creep
]
   Exit: (14) write(']') ? creep
   Call: (14) nl ? creep

   Exit: (14) nl ? creep
   Exit: (13) print_h(1, 3, 2) ? creep
   Call: (13) move(1, 2, 3, 1) ? creep
   Call: (14) _9504 is 1+ -1 ? creep
   Exit: (14) 0 is 1+ -1 ? creep
   Call: (14) move(0, 2, 1, 3) ? creep
   Exit: (14) move(0, 2, 1, 3) ? creep
   Call: (14) print_h(2, 3, 1) ? creep
   Call: (15) write(1) ? creep
1
   Exit: (15) write(1) ? creep
   Call: (15) write(' -> [') ? creep
 -> [
   Exit: (15) write(' -> [') ? creep
   Call: (15) write(2) ? creep
2
   Exit: (15) write(2) ? creep
   Call: (15) write(',') ? creep
,
   Exit: (15) write(',') ? creep
   Call: (15) write(3) ? creep
3
   Exit: (15) write(3) ? creep
   Call: (15) write(']') ? creep
]
   Exit: (15) write(']') ? creep
   Call: (15) nl ? creep

   Exit: (15) nl ? creep
   Exit: (14) print_h(2, 3, 1) ? creep
   Call: (14) move(0, 1, 3, 2) ? creep
   Exit: (14) move(0, 1, 3, 2) ? creep
   Exit: (13) move(1, 2, 3, 1) ? creep
   Exit: (12) move(2, 1, 3, 2) ? creep
   Call: (12) print_h(1, 2, 3) ? creep
   Call: (13) write(3) ? creep
3
   Exit: (13) write(3) ? creep
   Call: (13) write(' -> [') ? creep
 -> [
   Exit: (13) write(' -> [') ? creep
   Call: (13) write(1) ? creep
1
   Exit: (13) write(1) ? creep
   Call: (13) write(',') ? creep
,
   Exit: (13) write(',') ? creep
   Call: (13) write(2) ? creep
2
   Exit: (13) write(2) ? creep
   Call: (13) write(']') ? creep
]
   Exit: (13) write(']') ? creep
   Call: (13) nl ? creep

   Exit: (13) nl ? creep
   Exit: (12) print_h(1, 2, 3) ? creep
   Call: (12) move(2, 3, 2, 1) ? creep
   Call: (13) _11314 is 2+ -1 ? creep
   Exit: (13) 1 is 2+ -1 ? creep
   Call: (13) move(1, 3, 1, 2) ? creep
   Call: (14) _11452 is 1+ -1 ? creep
   Exit: (14) 0 is 1+ -1 ? creep
   Call: (14) move(0, 3, 2, 1) ? creep
   Exit: (14) move(0, 3, 2, 1) ? creep
   Call: (14) print_h(3, 1, 1) ? creep
   Call: (15) write(1) ? creep
1
   Exit: (15) write(1) ? creep
   Call: (15) write(' -> [') ? creep
 -> [
   Exit: (15) write(' -> [') ? creep
   Call: (15) write(3) ? creep
3
   Exit: (15) write(3) ? creep
   Call: (15) write(',') ? creep
,
   Exit: (15) write(',') ? creep
   Call: (15) write(1) ? creep
1
   Exit: (15) write(1) ? creep
   Call: (15) write(']') ? creep
]
   Exit: (15) write(']') ? creep
   Call: (15) nl ? creep

   Exit: (15) nl ? creep
   Exit: (14) print_h(3, 1, 1) ? creep
   Call: (14) move(0, 2, 1, 3) ? creep
   Exit: (14) move(0, 2, 1, 3) ? creep
   Exit: (13) move(1, 3, 1, 2) ? creep
   Call: (13) print_h(3, 2, 2) ? creep
   Call: (14) write(2) ? creep
2
   Exit: (14) write(2) ? creep
   Call: (14) write(' -> [') ? creep
 -> [
   Exit: (14) write(' -> [') ? creep
   Call: (14) write(3) ? creep
3
   Exit: (14) write(3) ? creep
   Call: (14) write(',') ? creep
,
   Exit: (14) write(',') ? creep
   Call: (14) write(2) ? creep
2
   Exit: (14) write(2) ? creep
   Call: (14) write(']') ? creep
]
   Exit: (14) write(']') ? creep
   Call: (14) nl ? creep

   Exit: (14) nl ? creep
   Exit: (13) print_h(3, 2, 2) ? creep
   Call: (13) move(1, 1, 2, 3) ? creep
   Call: (14) _13218 is 1+ -1 ? creep
   Exit: (14) 0 is 1+ -1 ? creep
   Call: (14) move(0, 1, 3, 2) ? creep
   Exit: (14) move(0, 1, 3, 2) ? creep
   Call: (14) print_h(1, 2, 1) ? creep
   Call: (15) write(1) ? creep
1
   Exit: (15) write(1) ? creep
   Call: (15) write(' -> [') ? creep
 -> [
   Exit: (15) write(' -> [') ? creep
   Call: (15) write(1) ? creep
1
   Exit: (15) write(1) ? creep
   Call: (15) write(',') ? creep
,
   Exit: (15) write(',') ? creep
   Call: (15) write(2) ? creep
2
   Exit: (15) write(2) ? creep
   Call: (15) write(']') ? creep
]
   Exit: (15) write(']') ? creep
   Call: (15) nl ? creep

   Exit: (15) nl ? creep
   Exit: (14) print_h(1, 2, 1) ? creep
   Call: (14) move(0, 3, 2, 1) ? creep
   Exit: (14) move(0, 3, 2, 1) ? creep
   Exit: (13) move(1, 1, 2, 3) ? creep
   Exit: (12) move(2, 3, 2, 1) ? creep
   Exit: (11) move(3, 1, 2, 3) ? creep
   Exit: (10) hanoi(3) ? creep
true.

\end{lstlisting}
\section{quickSort}

\subsection{code}
\begin{lstlisting}[escapeinside=``]
`% 해보려고 시도는 해 보았으나 실패했습니다.`
quickSort([A|B]) :- 
	partition(A,B,[],[]).

divide([A|list],[B|L],[P],R) :- 
	A =|= P,    
	divide(list,L,[P],R).
divide([A|list],[],[P],list) :-
	A = P.
\end{lstlisting}

\% 피봇을 중심으로 왼쪽, 오른쪽 리스트를 만드려고 했으나 빈 리스트 두개에 insert하는 방법
을 찾지를 못했습니다. 그리고는 partition이라는 재귀함수를 만들어 계속 pivot과 비교한 다음에
그 값를 왼쪽이라는 리스트와 오른쪽이라는 리스트에 계속 머지시키는 방법을 사용하려 했는데
그렇게 하는 것을 실패했습니다. 그리고 마지막으로 한 리스트 내에서 pivot보다
작은 값과 큰 값을 바꿔준 후 그러면 pivot을 중심으로 왼쪽과 오른쪽에 나눠져 있을 것이므로
이를 pivot을 중심으로 왼쪽과 오른쪽 list를 만들어주려고 했습니다. 그를 위해 만든 함수가 위쪽의 함수인데 이도 그냥 실패했습니다. 그리고 아래쪽의 partition함수는 위의 divide함수와 별개로
왼쪽 리스트의 헤드와 pivot을 비교하고, 또 그 나머지 list의 헤드를 비교하는 것을 재귀로
반복해서 값을 나눠보려했는데 잘 안되었습니다.

\begin{lstlisting}[escapeinside=``]
partition([P],[],[],[]). 
partition(P,[A|B],[H,L],R) :-
	H <= P, partition(P,B,L,R),
partition(P,[A|B],L,[H|R]) :-
	H > P, partition(P,B,L,R).
\end{lstlisting}
\subsection{trace}

\section{nQueen}
포기했습니다.
\subsection{code}
\subsection{trace}

\section{shortest path}
\begin{lstlisting}[escapeinside=``]
`% 해보려고 시도는 해 보았으나 실패했습니다.`
%fact
near(1,2,6). near(1,3,3).   `% 정해진 길을 작은 숫자를 기준으로 설정해주었습니다.`
near(2,1,6). near(2,3,2). 
near(2,4,5).
near(3,1,3). near(3,4,3).
near(3,5,4).
near(4,5,2). near(4,6,3).
near(5,6,5).

%rule
edge(X,Y) :- near(X,Y,_).   `%길이 양방향이 되도록 edge라는 함수를 설정해주었습니다.`
edge(X,Y) :- near(Y,X,_).

path(X,X) :- !. `% 자기가 자기를 가리킬 때는 cut하도록 설정`
path(X,Y) :- 
	near(X,Z,_),    `% 길이 Z를 경유하여 Y에 가도록 설정`
	path(Z,Y).
	
sp(X,Y) :- 
	findall(Y,path(X,Y),L), `% 가장 짧은 길 찾기 전에 일단 모든 경로를 찾아보려했습니다`
	write(L).
	
\end{lstlisting}
\subsection{code}
\subsection{trace}

\end{document}